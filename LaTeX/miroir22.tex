% Ce fichier doit être en UTF-8 sans BOM (TeXworks ne tolère pas le BOM)
% Après ce préambule, les seuls caractères qui ne doivent pas être dans
% le texte du document sont: "\", "{", "}"
\documentclass[12pt]{article}
\usepackage[letterpaper,margin=2cm]{geometry}
\usepackage[utf8]{inputenc}
\usepackage[T1]{fontenc}
\usepackage[normalem]{ulem}
\usepackage{textcomp}
\usepackage[french]{babel}
\selectlanguage{french}
\usepackage{graphicx}
\makeatletter
\def\maxwidth#1{\ifdim\Gin@nat@width>#1 #1\else\Gin@nat@width\fi}
\makeatother
\newcommand{\image}[1]{\includegraphics[width=\maxwidth{\textwidth}]{#1}}
%\newcommand{\scr}{\fontfamily{pzc}\selectfont}
%\newcommand{\ital}{\sf\em}
\catcode`\&=\active
\def&{\&}
%\catcode160=\active
%\def {~}
\catcode`_=11		% lettre
%\def_{\_}
\setlength{\parindent}{0cm}
\setlength{\parskip}{0.8\baselineskip}
\newcommand{\hr}{\rule{\textwidth}{0.4pt}}
\begin{document}
\renewcommand{\labelitemi}{$\bullet$}	% Doit être après le \begin{document}
\renewcommand{\labelitemii}{$\circ$}	% Doit être après le \begin{document}
\renewcommand{\labelitemiii}{-}	% Doit être après le \begin{document}
\catcode`$=\active		% Doit être après les $\bullet$ etc.
\def${\$}
\catcode`#=\active
\def#{\#}
% Ce qui suit doit être après le \begin{document}, sinon erreur ésotérique.
\catcode`\%=\active
\def%{\%}

     {\bf \huge Code d’identification : miroir22} \\
    \newline \hr \begin{center} 
        \image{../photos/miroir22.jpg}
       
    \end{center} 
    \begin{itemize}
    
    \footnotesize {\item {\bf Date de création de la fiche :} 2021-02-03
    {par Céline Hostiou}} 
    \end{itemize}
   \hr 
    \section* {Aspect} 
    {\bf \large Couleur :} argent
    \\ \\ {\bf \large Forme :} rond 
    \\ \\ {\bf \large Style :} 
            ce miroir rond date des années 1960. Il a pour matériaux le verre et
            le cristal.
         
        {\bf \large Époque :} XXᵉ siècle 
    \\ \\ {\bf \large Cadre :} le cadre est gravé et biseauté.
         
    \section* {Mesures}
     {\bf \large Hauteur :} 62 cm
   \\ \\ {\bf \large Largeur :} 62 cm 
  
    \section* {Origine}
    {\bf \large Date d’acquisition :} 2017-12-15 \\ \\
    {\bf \large Moyen d’acquisition :} achat \\ \\ {\bf \large Prix d’achat :}
        250 € 
      
    \subsection* {Commentaire}
     Ce miroir circulaire, comme le 
    \href{miroir50.xml}{miroir 50}
  ,
                a été acquis auprès d’un brocanteur de Saint-Ouen, situé près de Paris. Bien que
                d’occasion, il ne présente ni fêlés ni impacts. Il est visible sous ce 
    \href{https://www.selency.fr/produit/55V2Q3Y2/miroir-rond-grave-et-biseaute-annees-60.html}{lien}
  . 
    \section* {Valeur actuelle}
    En mai 2020, le prix du miroir était de 250 €.
    \section* {Usage attendu}
     Ce miroir peut être fixé dans n’importe quelle pièce du logement, dans un vestibule, un
            couloir.  \end{document} 
  