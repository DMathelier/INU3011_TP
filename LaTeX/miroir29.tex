 
    % Ce fichier doit être en UTF-8 sans BOM (TeXworks ne tolère pas le BOM)
% Après ce préambule, les seuls caractères qui ne doivent pas être dans
% le texte du document sont: "\", "{", "}"
\documentclass[12pt]{article}
\usepackage[letterpaper,margin=2cm]{geometry}
\usepackage[utf8]{inputenc}
\usepackage[T1]{fontenc}
\usepackage[normalem]{ulem}
\usepackage{textcomp}
\usepackage[french]{babel}
\selectlanguage{french}
\usepackage{graphicx}
\makeatletter
\def\maxwidth#1{\ifdim\Gin@nat@width>#1 #1\else\Gin@nat@width\fi}
\makeatother
\newcommand{\image}[1]{\includegraphics[width=\maxwidth{\textwidth}]{#1}}
%\newcommand{\scr}{\fontfamily{pzc}\selectfont}
%\newcommand{\ital}{\sf\em}
\catcode`\&=\active
\def&{\&}
%\catcode160=\active
%\def {~}
\catcode`_=11		% lettre
%\def_{\_}
\setlength{\parindent}{0cm}
\setlength{\parskip}{0.8\baselineskip}
\newcommand{\hr}{\rule{\textwidth}{0.4pt}}
\begin{document}
\renewcommand{\labelitemi}{$\bullet$}	% Doit être après le \begin{document}
\renewcommand{\labelitemii}{$\circ$}	% Doit être après le \begin{document}
\renewcommand{\labelitemiii}{-}	% Doit être après le \begin{document}
\catcode`$=\active		% Doit être après les $\bullet$ etc.
\def${\$}
\catcode`#=\active
\def#{\#}
% Ce qui suit doit être après le \begin{document}, sinon erreur ésotérique.
\catcode`\%=\active
\def%{\%}

    
    {\bf \huge Code d’identification : miroir29}
    \\ \\ \hr
    \begin{center}
    \image{../photos/miroir29.jpg}
    \end{center}
    
    {\bf \large Date de création de la fiche :} 2021-02-20
    {par Céline Hostiou}
  
    {\bf \large Date de modification de la fiche :} 
    {par 2021-03-01}
    Daphné Mathelier \\ \\ \hr
    \section* {Aspect}
   {\bf \large Couleur :} noir
    \\ \\ {\bf \large Forme :}  rectangle
    \\ \\ {\bf \large Style :} 
            ce miroir rectangulaire dit « Karmsund », vendu par la chaîne de
            magasins Ikea, se déploie en triptyque. Il dispose de pieds et est coiffé d’une
            ornementation, tous en fer forgé, tous détachables.
        
        {\bf \large Époque :} contemporain
    \\ \\ {\bf \large Cadre :} le cadre est en acier galvanisé noir.
        
    \section* {Mesures}
      
  {\bf \large Hauteur :} 80 cm
   
   {\bf \large Largeur :} 74 cm
    
   {\bf \large Profondeur :} 6 cm
    
   {\bf \large Poids :} 7,10 kg
     
    \section* {Origine}
    {\bf \large Date d’acquisition :} 2018-05-12

    {\bf \large Moyen d’acquisition :} achat
          \\ \\{\bf \large Prix d’achat :} 33 €
        
      \subsection* {Commentaire}
       
    
Ce miroir rectangulaire, comme le 
    \href{miroir42.xml}{miroir
                    42}
  , a été acheté chez Ikea. La fiche de ce produit est consultable
                sous ce  
    \href{https://www.ikea.com/ca/fr/p/karmsund-miroir-de-table-noir-20294983/}{lien}
  . 
    

      \section* {Valeur actuelle} En mai 2020, le prix du miroir était de 50 $CAN (33 €).
      \section* {Usage attendu}  
    
Des crochets situés derrière les miroirs latéraux permettent de ranger facilement ses
        bijoux. Les sections pivotantes du miroir permettent de s’admirer sous différents angles.
        Peut être installé dans toutes les pièces de la maison, dont la salle de bains, car il a été
        testé et approuvé pour cet usage. 
    

    \end{document}
  