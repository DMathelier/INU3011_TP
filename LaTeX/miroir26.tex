% Ce fichier doit être en UTF-8 sans BOM (TeXworks ne tolère pas le BOM)
% Après ce préambule, les seuls caractères qui ne doivent pas être dans
% le texte du document sont: "\", "{", "}"
\documentclass[12pt]{article}
\usepackage[letterpaper,margin=2cm]{geometry}
\usepackage[utf8]{inputenc}
\usepackage[T1]{fontenc}
\usepackage[normalem]{ulem}
\usepackage{textcomp}
\usepackage[french]{babel}
\selectlanguage{french}
\usepackage{graphicx}
\makeatletter
\def\maxwidth#1{\ifdim\Gin@nat@width>#1 #1\else\Gin@nat@width\fi}
\makeatother
\newcommand{\image}[1]{\includegraphics[width=\maxwidth{\textwidth}]{#1}}
%\newcommand{\scr}{\fontfamily{pzc}\selectfont}
%\newcommand{\ital}{\sf\em}
\catcode`\&=\active
\def&{\&}
%\catcode160=\active
%\def {~}
\catcode`_=11		% lettre
%\def_{\_}
\setlength{\parindent}{0cm}
\setlength{\parskip}{0.8\baselineskip}
\newcommand{\hr}{\rule{\textwidth}{0.4pt}}
\begin{document}
\renewcommand{\labelitemi}{$\bullet$}	% Doit être après le \begin{document}
\renewcommand{\labelitemii}{$\circ$}	% Doit être après le \begin{document}
\renewcommand{\labelitemiii}{-}	% Doit être après le \begin{document}
\catcode`$=\active		% Doit être après les $\bullet$ etc.
\def${\$}
\catcode`#=\active
\def#{\#}
% Ce qui suit doit être après le \begin{document}, sinon erreur ésotérique.
\catcode`\%=\active
\def%{\%}

     {\bf \huge Code d’identification : miroir26} \\
    \newline \hr \begin{center} 
        \image{../photos/miroir26.jpg}
       
    \end{center} 
    \begin{itemize}
    
    \footnotesize {\item {\bf Date de création de la fiche :} 2021-02-12
    {par Céline Hostiou}} 
    \footnotesize {\item {\bf Date de modification de la fiche :} 2021-03-01
    {par Daphné Mathelier} \\}
    \end{itemize}
   \hr 
    \section* {Aspect} 
    {\bf \large Couleur :} noir
    \\ \\ {\bf \large Forme :} triangle 
    \\ \\ {\bf \large Style :} 
            ce miroir triangulaire de marque « Tikamoon » et de modèle « Maho »
                est rehaussé de quatre barres superposées, ouvragées. Celles-ci épousent l’angle supérieur du triangle.
         
        {\bf \large Époque :} contemporain 
    \\ \\ {\bf \large Cadre :} le cadre, en métal peint, apporte une touche industrielle et graphique à la pièce où il est exposé. 
    \section* {Mesures}
     {\bf \large Hauteur :} 60 cm
   \\ \\ {\bf \large Largeur :} 60 cm 
   \\ \\ {\bf \large Profondeur :} 4 cm  
   \\ \\ {\bf \large Poids :} 8 kg
  
    \section* {Origine}
    {\bf \large Date d’acquisition :} 2018-02-01 \\ \\
    {\bf \large Moyen d’acquisition :} achat \\ \\ {\bf \large Prix d’achat :}
        149 € 
      
    \subsection* {Commentaire}
     Ce miroir triangulaire, comme le 
    \href{miroir38.xml}{miroir
                    38}
  , a été conçu par l’entreprise Tikamoon, située dans le nord de
                la France, à Lille. Celle-ci s’inscrit dans un créneau de meubles durables. Elle a
                développé une chaîne de production à l’impact environnemental limité.  La fiche produit du miroir est consultable sous ce 
    \href{https://www.tikamoon.com/art-miroir-triangulaire-en-m-tal-60-maho-2782.htm}{lien}
  . 
    \section* {Valeur actuelle}
    En mai 2020, le prix du miroir était de 149 €.
    \section* {Usage attendu}
     Ce miroir peut agrémenté toutes les pièces de la maison.  \end{document} 
  